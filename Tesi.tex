\documentclass[a4paper,12pt,twoside]{book}
\usepackage{graphicx}
\usepackage{fancyhdr}
\usepackage[font = scriptsize, bf]{caption}
\usepackage[english]{babel}
\usepackage[utf8x]{inputenc}
\usepackage[parfill]{parskip}
\usepackage{amsmath, amssymb}
\usepackage{moreverb}
\usepackage{algorithm} %
\usepackage{algpseudocode}
\usepackage[usenames,dvipsnames]{color}
\usepackage[swapnames]{frontespizio}
\usepackage{url}
\usepackage{setspace}
\usepackage{eqparbox,array}
\usepackage{siunitx}
\usepackage{subfigure} 
\usepackage{wrapfig}
\usepackage{amsthm}
\usepackage{eurosym}

\renewcommand{\algorithmiccomment}[1]{  //\emph{\textcolor{Gray}{#1}}}


% Sistema i margini per lasciare più spazio di rilegatura
\addtolength{\oddsidemargin}{+1,3cm} 
\addtolength{\evensidemargin}{-1,3cm} 
\onehalfspacing

% Imposta lo stile della prima pagina del capitolo
\fancypagestyle{plain} {
    \fancyhead{}
    \fancyfoot[LE,RO]{\thepage}
    \renewcommand{\headrulewidth}{0pt}
}

\DeclareMathOperator*{\argmax}{arg\,max}
\newcommand{\compInterfacciaDB}{Data Interface}
\newcommand{\compLoader}{Loader}
\newcommand{\compMatrix}{Matrix Creator}
\newcommand{\compTermsSel}{Terms Selector}
\newcommand{\compPosition}{Position Calculator}
\newcommand{\compClustering}{Clustering Component}
\newcommand{\compEvolution}{Evolution Discoverer}

% \newtheorem{thm}[equation]{Theorem}
\newtheorem{thm}{Theorem}[chapter]
\newtheorem{exmp}{Example}[chapter]
\newtheorem{defn}{Definition}[chapter]
\newtheorem{prp}{Property}[chapter]



\hyphenation{ti-me-win-dow}

\graphicspath{{./Immagini/}}


\begin{document}

% frontespizio
\begin{frontespizio}
  \Istituzione{University of Bari ``Aldo Moro''}
  \Logo[3.5cm]{Immagini/logo_uni.jpeg}
  \Divisione{Department of Computer Science}
  \Scuola{Master's Degree in Computer Science}
  \Titolo{Argument Mining}
  \Sottotitolo{Artificial Intelligence}
  \NCandidato{Student}
  \Candidato{Christopher Piemonte}
  \NRelatore{Professor}{Professors}
  \Relatore{Stefano Ferilli}
  \Relatore{Andrea Pazienza}
  \Piede{Academic year 2017-2018}
\end{frontespizio}

% \IfFileExists{Tesi-frn.pdf}{}{%
% 	\immediate\write18{pdflatex Tesi-frn}
% }

\IfFileExists{\jobname-frn.pdf}{}{%
\immediate\write18{pdflatex \jobname-frn}}

\pagestyle{fancy}
\fancyfoot{}
\fancyfoot[LE,RO]{\thepage}
\fancyhead{}
\renewcommand{\headrulewidth}{0pt}
\headheight = 15pt
\frontmatter

% indice
\tableofcontents
\afterpage{\null\newpage}


\frontmatter	
\mainmatter

% Imposta lo stile di intestazione e piè di pagina dei capitoli
\fancyfoot{}
\fancyhead{}
\fancyhead[LE,RO]{\slshape \leftmark}
\fancyfoot[LE,RO]{\thepage}
\renewcommand{\headrulewidth}{1pt}
\renewcommand{\chaptermark}[1]{%
\markboth{\thechapter.\ #1}{}}


% !TEX encoding = UTF-8
% !TEX TS-program = pdflatex
% !TEX root = ../Tesi.tex
% !TEX spellcheck = it-IT

%*******************************************************
% Introduzione
%*******************************************************
\cleardoublepage
\chapter*{Introduzione}
L'argomentazione rappresenta un approccio al ragionamento nei casi in cui si dispone di conoscenza inconsistente, e può essere considerata come un metodo per gestire l’incertezza. Infatti, l’idea alla base dell'argomentazione è quella di valutare il motivo per cui un fatto sia considerato vero analizzando gli argomenti e le relazioni che intercorrono tra essi per valutarne la certezza. Tale processo può essere visto come una forma di ragionamento riguardo gli argomenti per determinarne i più accettabili. Sebbene il termine argomentare possa intuitivamente richiamare diversi significati come quello del ragionamento a partire da premesse fino alle conclusioni o l'esprimere la propria opinione in una discussione, un argomento non si lega a particolari strutture ma in senso astratto è qualsiasi cosa che può attaccare o essere attaccata da un altro argomento. Per tale motivo, un Argumentation Framework può essere adeguato a rappresentare diverse situazioni. La possibilità dell'Argumentation Framework di poter rappresentare diverse situazioni ha portato, nel tempo, alla proposta di estensioni che ponessero attenzione su diversi aspetti, come i tipi di relazione che possono sussistere o la quantificazione della forza di una relazione tra due argomenti.

La capacità di argomentare costituisce una caratteristica fondamentale nel discorso umano. Sia discutendo con altre persone che scrivendo un commento sui Social Media, gli argomenti sono onnipresenti, nella vita reale tanto quanto nel World Wide Web. L'Argumentation Theory è un ramo dell'Intelligenza Artificiale finalizzato a fornire un modello formale per la rappresentazione, la costruzione e la semantica delle argomentazioni. Nella sua forma più semplice, l’Argumentation Framework (AF) di Dung, consiste nel determinare insiemi coerenti di argomenti all'interno di un grafo, composto da entità astratte (gli argomenti) e relazioni binarie di attacco tra queste, fornendo un insieme di semantiche per verificare particolari proprietà ad esso relative \cite{dung1995acceptability}. 

Problema fondamentale risulta quindi la costruzione automatica di argomenti dalle fonti dati disponibili, che possono essere più o meno strutturate. La quantità crescente di dati sul web implica che l'analisi manuale di questi contenuti, inclusi dibattiti e argomenti, è diventata ormai irrealizzabile. L'Argument Mining affronta questo problema sviluppando soluzioni che automatizzano, o almeno facilitano, il processo di costruzione di Argument Framework da testo libero. Per costruire AF ci interessano generalmente due problemi:

\begin{itemize}
    \item l'identificazione degli argomenti
    \item l'identificazione delle relazioni tra gli argomenti.
\end{itemize}

L'Argument Mining è un compito complesso in quanto il linguaggio naturale non presenta una struttura facilmente individuabile.

% da aggiungere parti relative a impl e lavoro

\chapter{Argumentation Theory}
\label{cap:capitolo1}
% !TEX encoding = UTF-8
% !TEX TS-program = pdflatex
% !TEX root = ../Tesi.tex

%************************************************

%************************************************


\section{Argumentation}



\chapter{Mining dai Social Media}
\label{cap:capitolo2}
% !TEX encoding = UTF-8
% !TEX TS-program = pdflatex
% !TEX root = ../Tesi.tex

%************************************************

%************************************************
Segue un overview del metodo generale utilizzato, in sezione \ref{section:approcci} e degli approcci specifici, in sezione \ref{section:data_sources}, per ogni Social Media in esame da cui è stato estratto un argumentation framework.

\section{Metodologie di Mining}
\label{section:approcci}
L'estrazione di argomenti prende in esame la struttura che i commenti in uno specifico Social Network possono assumere. In tutte le fonti analizzate un commento può considerarsi una risposta ad un solo altro commento, questo restituisce una struttura ad albero che, partendo dal nodo radice, si dirama attraverso relazioni di risposta. Di seguito viene descritta l'estrazione dell'albero dei commenti in sezione \ref{subsection:albero_commenti} ed un tentativo di estrazione di un grafo considerando come nodi gli utenti in sezione \ref{subsection:grafo_utenti}.

\subsection{Albero dei commenti}
\label{subsection:albero_commenti}
Per la costruzione dell'albero dei commenti serve costruire un grafo orientato della conversazione $\mathcal{G = ⟨V, E⟩}$, dove i nodi $v \in \mathcal{V}$ sono commenti ed esiste un arco $(v_1,v_2) \in \mathcal{E}$ se il commento $v_1$ risponde al commento $v_2$.

La struttura dell'albero può variare in base alla sorgente dei dati in esame, ad esempio alcuni Social Network pongono un limite al numero di livelli dell'albero, principalmente per motivi di presentazione.
Estrarre delle semantiche da Argumentation Framework ricavati dall'albero della conversazione equivale ad estrarre i commenti che soddisfano i requisiti delle varie semantiche.

\subsection{Grafo degli utenti}
\label{subsection:grafo_utenti}
Invece di considerare i commenti come argomenti, può essere considerato argomento ogni utente nella conversazione, e creare archi nel Argumentation Framework risultante tra due utenti se un utente ha risposto almeno una volta ad un altro utente. Più formalmente costruire un grafo orientato $\mathcal{G = ⟨V, E⟩}$, dove i nodi $v \in \mathcal{V}$ sono utenti ed esiste un arco $(v_1,v_2) \in \mathcal{E}$ se l'utente $v_1$ ha risposto almeno una volta all'utente $v_2$.

In questo modo se ogni commento nell'albero della conversazione ha un utente univoco, l'Argumentation Framework risultante sarà comunque un albero, mentre se un utente risponde ad almeno due (o più) utenti il suo nodo nell'$AF$ avrà due (o più) archi uscenti, e se più commenti dello stesso utente vengono risposti da due (o più) utenti, il suo nodo nell'$AF$ avrà due (o più) archi entranti.
Estrarre delle semantiche da Argumentation Framework ricavati dal grafo degli utenti equivale ad estrarre gli utenti che soddisfano i requisiti delle varie semantiche.

\subsection{Pesi delle relazioni}
\label{subsection:weight}
Le relazioni di risposta nell'albero dei commenti vengono pesate in base alla combinazione di similarità e sentiment dei commenti della relazione.

\subsubsection{Similarity}
La \textit{similarity} è ricavata attraverso l'embedding dei testi dei commenti ricavati da un modello di Word2Vec pre-addestrato sulle news di Google \cite{googlenewsmodel}. Gli Word Embedding sono un insieme di tecniche per la rappresentazione vettoriale delle parole di un vocabolario, introdotti per la prima volta in \cite{mikolov2013distributed}, i quali sono capaci di catturare il contesto delle parole all'interno di un documento attraverso la \textit{distributional semantics}. I vettori risultanti codificano proprietà linguistiche nello spazio vettoriale e nella distanza tra di essi, ad esempio è possibile fare analogie come mostrato in figura \ref{fig:analogy}.

\begin{figure}
    \includegraphics[width=\linewidth]{Immagini/king-queen.png}
    \caption{Analogie tra vettori.}
    \label{fig:analogy}
\end{figure}

L'emebedding dei testi dei commenti è ottenuto facendo la media degli embedding delle parole contenute. Infine la similarità tra due testi $c_1$ e $c_2$ si ricava con:

$$cosineSimilarity(c_1, c_2) = \frac{c_1 \cdot c_2}{||c_1|| ||c_2||}$$

che restituirà quindi un risultato compreso $\in [0, 1]$.

\subsubsection{Sentiment}
La \textit{sentiment analysis} si riferisce all'utilizzo dell'elaborazione del linguaggio naturale per identificare, estrarre e quantificare sistematicamente informazioni soggettive. L'analisi del sentiment è ampiamente applicata per analizzare social media per una varietà di applicazioni. Il valore è ricavato attraverso VADER \cite{hutto2014vader} uno strumento di analisi del sentimento basato su regole che è ideato specificamente per individuare la polarità nei testi espressi nei social media, ad esempio:

    $$sentiment(\ "\textbf{:)}" \ ) = 0.4588$$
    $$sentiment(\ "\textbf{:(}" \ ) = -0.4404$$

che restituirà quindi un risultato compreso $\in [-1, 1]$.

\subsubsection{Peso finale}
Il peso finale dell'arco è ottenuto prendendo in considerazione la similarità tra i commenti, in modo da pesare di più commenti che parlano dello stesso argomento e di meno commenti che parlano di argomenti diversi, ed il sentiment in modo da capire se i commenti sono in accordo o in disaccordo attraverso:

$$weight = similarity_{c_1 c_2} \cdot sentiment_{c_1} \cdot sentiment_{c_2}$$

che restituirà quindi un risultato compreso $\in [-1, 1]$.

\section{Analisi dei Social Media}
\label{section:data_sources}

\subsection {Twitter} %%%%%%%%%%%%%%%%%%%%%%%%%%%%%%%%%%%%%%%%%%%

Lo scraping del sito Web è vietato dal Twitter Terms of Service, quindi l'estrazione della conversazione avviene tramite API. Per accedere alla piattaforma delle API di twitter è necessario creare un Twitter developer account ed in seguito creare dal portale una applicazione nella quale si dichiara lo scopo dello dell'accesso ai contenuti e si richiedono i relativi permessi. Una volta ottenuta l'approvazione, vengono rilasciati delle credenziali per l'autenticazione agli endpoint. Gli step necessari sono:
  

\begin{itemize}  
    \item Andare su {\color{blue}\underline{\href{https://.dev.twitter.com}{Twitter Developers Site}}}
    \item Login con account di Twitter
    \item Andare su "My Applications"
    \item Creare una nuova applicazione
    \item Compilare con i dati richiesti
    \item Creare l'Access Token
    \item Scegliere il tipo di permessi
    \item Prendere nota delle impostazioni OAuth
    \item Infine, tutto il necessario per l'estrazione della conversazione è: \begin{itemize}
        \item Consumer key
        \item Consumer secret
        \item Access token
        \item Access token secret
    \end{itemize}
\end{itemize}

I contenuti messi a disposizione sono i Tweet pubblici, reperiti cercando parole chiave specifiche attraverso le Search API o chiedendo esempi di Tweet a determinati account attraverso le User API. Inoltre è possibile richiedere tweet specifici conoscendone l'id.

Al momento non è disponibile l'accesso alle conversazioni, la richiesta di tweet tramite le Search API o le User API restituisce solo un sottoinsieme dei tweet ed inoltre c'è un limite al numero di richieste giornaliere. Questo rende difficile cercare di ricostruire le conversazioni dai tweet restituiti dalla query. 

\subsubsection{Ricostruire la conversazione}
\label{ricostruire-conv}
La richiesta ad un determinato topic restituisce un campione di massimo 1000 tweet recenti.
Fra gli attributi presenti nei tweet è presente il campo \textbf{"in reply to status id"} che, se presente, contiene l'id del tweet al quale il tweet corrente sta rispondendo. Questo id può essere utilizzato per cercare di ricostruire la conversazione, facendo una richiesta tramite API per avere il tweet risposto e, se questo a sua volta risponde ad un altro tweet, continuare a seguire la catena di risposte (la conversazione). Per qualsiasi tweet con questo campo, possiamo:
\begin{itemize}
    \item trovare il tweet a cui quello corrente sta rispondendo, quindi ripetere la procedura per ogni tweet con questo attributo avvalorato in modo da ottenere delle sequenze di tweet.
    \item Una volta create le sequenze di tweet, se queste contengono elementi in comune, unire le sequenze creando un grafo della conversazione.
    \item Valutare la bontà della conversazione ottenuta mediante euristiche come ad esempio il numero di nodi (tweet) nel grafo della conversazione, la presenza di diramazioni o il numero di utenti distinti, in modo da poter scegliere la conversazione "migliore".
\end{itemize}

Quindi costruire grafo orientato e pesato della conversazione $\mathcal{G = ⟨V, E⟩}$, dove i nodi $v \in \mathcal{V}$ sono tweet ed esiste un arco $(v_1,v_2) \in \mathcal{E}$ se l'attributo \textbf{"in reply to status id"} di $v_1$ contiene l'id di $v_2$, pesato come descritto in \ref{subsection:weight}.

Tuttavia le API forniscono l'accesso solo ad un campione dei tweet, quindi potrebbe non essere possibile recuperare dei tweet quando si tenta di ricostruire la conversazione. Un'altra limitazione proviene dal numero di tweet iniziale che è possibile ottenere in risposta alla query, che risulta essere massimo 1000, dai quali è necessario filtrare i tweet che non contengono il campo \textbf{"in reply to status id"} avvalorato(di solito circa il 80\%). Inoltre i risultati ottenuti cambiano in modo notevole al variare della query e del momento in cui viene effettuata la richiesta. La maggior parte delle volte le conversazioni risultano essere sequenze lineari di nodi, ovvero grafi con archi in entrata $\leq 1$ ed archi in uscita $\leq 1$ come rappresentato in figura \ref{fig:comment-twitter}.

\begin{figure}[ht]
    \includegraphics[width=\linewidth]{Immagini/twitter.png}
    \caption{Rappresentazione del grafo della conversazione, utilizzando come nodi i commenti.}
    \label{fig:comment-twitter}
\end{figure}

\subsubsection {Utenti come nodi}
Un altro approccio è quello descritto in sezione \ref{subsection:grafo_utenti}, ma con la differenza che qui non c'è l'albero della conversazione ma ci sono i 1000 tweet restituiti dalla richiesta API più eventuali altri tweet ottenuti seguendo il campo \textbf{"in reply to status id"}. Quindi viene costruito il grafo orientato pesato $\mathcal{G = ⟨V, E⟩}$, dove i nodi $v \in \mathcal{V}$ sono utenti ed esiste un arco $(v_1,v_2) \in \mathcal{E}$ se esiste almeno un tweet con utente $v_1$ e con l'attributo \textbf{"in reply to status id"} contenente l'id di un tweet appartenente all'utente $v_2$, pesato come descritto in \ref{subsection:weight}.

Anche qui si ripropongono i problemi presentati nella sezione precedente \ref{ricostruire-conv}, tuttavia il grafo risultante presenta nodi con archi in entrata $\geq 1$ ed archi in uscita $\geq 1$, come rappresentato in figura \ref{fig:users-twitter}.

\begin{figure}[H]
    \includegraphics[width=\linewidth]{Immagini/twitter-users.png}
    \caption{Rappresentazione del grafo della conversazione, utilizzando come nodi gli utenti.}
    \label{fig:users-twitter}
\end{figure}



%%%%%%%%%%%%%%%%%%%%%%%%%%%%%%%%%%%%%%%%%%%%%%%%%%%%%%%%%%%%%%%
%%%%%%%%%%%%%%%%%%%%%%%%%%%%%%%%%%%%%%%%%%%%%%%%%%%%%%%%%%%%%%%
%%%%%%%%%%%%%%%%%%%%%%%%%%%%%%%%%%%%%%%%%%%%%%%%%%%%%%%%%%%%%%%
%%%%%%%%%%%%%%%%%%%%%%%%%%%%%%%%%%%%%%%%%%%%%%%%%%%%%%%%%%%%%%%
%%%%%%%%%%%%%%%%%%%%%%%%%%%%%%%%%%%%%%%%%%%%%%%%%%%%%%%%%%%%%%%


\subsection {Stackoverflow} %%%%%%%%%%%%%%%%%%%%%%%%%%%%%%%%%%%%%%%%%%%

\begin{itemize}
    \item create account at https://stackapps.com/
    \item register an app
    \item b
\end{itemize}

\subsubsection {Overview}
\subsubsection {API}

\subsubsection {Commenti}
\subsubsection {Utenti}


%%%%%%%%%%%%%%%%%%%%%%%%%%%%%%%%%%%%%%%%%%%%%%%%%%%%%%%%%%%%%%%
%%%%%%%%%%%%%%%%%%%%%%%%%%%%%%%%%%%%%%%%%%%%%%%%%%%%%%%%%%%%%%%
%%%%%%%%%%%%%%%%%%%%%%%%%%%%%%%%%%%%%%%%%%%%%%%%%%%%%%%%%%%%%%%
%%%%%%%%%%%%%%%%%%%%%%%%%%%%%%%%%%%%%%%%%%%%%%%%%%%%%%%%%%%%%%%
%%%%%%%%%%%%%%%%%%%%%%%%%%%%%%%%%%%%%%%%%%%%%%%%%%%%%%%%%%%%%%%

\subsection {Reddit} %%%%%%%%%%%%%%%%%%%%%%%%%%%%%%%%%%%%%%%%%%%

\begin{itemize}
    \item create account at 
    \item register an app
    \item b
\end{itemize}


\subsubsection {Overview}
\subsubsection {API}

\subsubsection {Commenti}
\subsubsection {Utenti}

% Fatti generati argument/1, attac- k/2, support/2, rel_weight/3

% descrizione vader e altri componenti utilizzati



\chapter{Sperimentazione}
\label{cap:capitolo3}
% !TEX encoding = UTF-8
% !TEX TS-program = pdflatex
% !TEX root = ../Tesi.tex
% !TEX spellcheck = it-IT

%************************************************

%************************************************

% ranking extensions


\section {Generazione estensioni}
La generazione delle estensioni all'interno di un grafo di argomenti equivale all'estrarre sottoinsiemi di nodi coerenti fra di loro secondo i vincoli imposta dalla semantica. Il mining dei social media permette di mappare in questo dominio i commenti o gli utenti, estraendo quindi sottoinsiemi di questi che soddisfano i requisiti. 

Dipendentemente dal tipo di conversazione estratta i gruppi di commenti/utenti coerenti fra di loro possono avere diverse interpretazioni. All'interno di un dibattito politico potrebbero rappresentare l'accordo o il disaccordo con un certo argomento portato dal politico, in un post su di una serie tv potrebbe rappresentare il gradimento verso un particolare personaggio o sottotrama. 

\subsection {Implementazione}
Per l'implementazione delle estensioni è stato utilizzato il framework \textit{Arguer} \cite{}. Il sistema implementato in Prolog permette di importare Argumentation Framework sotto forma di basi di conoscenza prolog e calcolare le semantiche. 

È stato scelto di utilizzare il sistema Arguer in quanto era già presente come framework per il calcolo delle semantiche e l'implementazione di un tale sistema da zero avrebbe richiesto lo l'impegno tale da costituire un progetto a se stante.

%%%%%%%%%%%%%%%%%%%%%%%%%%%%%%%%%%%%%%%%%%%%%%%%%%%%%%%%%%%%%%%
%%%%%%%%%%%%%%%%%%%%%%%%%%%%%%%%%%%%%%%%%%%%%%%%%%%%%%%%%%%%%%%
%%%%%%%%%%%%%%%%%%%%%%%%%%%%%%%%%%%%%%%%%%%%%%%%%%%%%%%%%%%%%%%
%%%%%%%%%%%%%%%%%%%%%%%%%%%%%%%%%%%%%%%%%%%%%%%%%%%%%%%%%%%%%%%
%%%%%%%%%%%%%%%%%%%%%%%%%%%%%%%%%%%%%%%%%%%%%%%%%%%%%%%%%%%%%%%

\section {Ranking delle estensioni}
La generazione delle semantiche si limita a verificare i requisiti imposti dalla semantica ed a restituire un insieme di insiemi di nodi. Tutti i sottoinsiemi che soddisfano i vincoli sono egualmente importante. Collocando l'interpretazione degli argomenti in un dato dominio e guidati da una determinata applicazione, potrebbe essere utile dare una priorità a certe estensioni in base a delle metriche dei suoi nodi ad esempio o alla lunghezza del sottoinsieme. 

Le metriche possono prendere in considerazione diversi aspetti dell argumentation framework, dalla topologia del grafo al testo del commento. Fra le graph based quelle implementate sono:

\begin{itemize}
    \item \textbf{Distanza fra i nodi del sottoinsieme}: calcolata come la distanza fra tutte le possibili coppie all'interno della estensione (divisa per la lunghezza della estensione).

    \item \textbf{indegree nel grafo di difesa}: un nodo $\mathcal{V}$ è difeso da un altro nodo $\mathcal{U}$ se esiste un altro nodo $\mathcal{W}$ tale che $\mathcal{W}$ attacca $\mathcal{V}$ e $\mathcal{U}$ attacca $\mathcal{W}$. Mentre per BAF e BWAF oltre la relazione di difesa è considerata anche la relazione di supporto.

    \item \textbf{closeness nel grafo di difesa}: valore attribuito ad un nodo calcolando la media di tutti gli shortest path con sorgente quel nodo e destinazione gli altri nodi.

    \item \textbf{indegree nel grafo di attacchi}: per questa metrica vengono considerati solo le relazione di attacco, andando a considerare migliori i nodi con indegree basso.

    \item \textbf{closeness nel grafo di attacchi}: stesso valore descritto sopra ma calcolato nel grafo considerando solo le relazioni di attacco.
\end{itemize}


Il valore aggiunto di una metrica è da attribuire a logiche di business in base alla specifica interpretazione e conoscenza che si ha degli argomenti.



\chapter{Critical analysis}
\label{cap:capitolo4}
% !TEX encoding = UTF-8
% !TEX TS-program = pdflatex
% !TEX root = ../Tesi.tex
% !TEX spellcheck = it-IT

%************************************************

%************************************************


Per l'utilizzo del modulo python di mining è disponibile un repository github all'indirizzo \url{https://github.com/chrisPiemonte/argonaut} contenente un README.md dove viene spiegato l'utilizzo e l'installazione del modulo. Per la parte di mining è disponibile uno script da configurare cambiando i parametri in base alla fonte dati da cui attingere per costruire il grafo di argomenti.

Per il calcolo delle estensioni ed i loro ranking tramite il sistema prolog invece, il repository contiene tutto il necessario per il ranking delle estensioni tranne però il sistema Arguer per la generazione delle estensioni, che deve essere reperito a parte. Per l'utilizzo basta lanciare il file in \textbf{src/reasoner/argonaut} ed eseguire il predicato \textbf{argonaut}. Comparirà da linea di comando un menù per il caricamento, la generazione delle estensioni e per la loro valutazione.



Nel presente lavoro è stato proposto un sistema di argumentation mining in grado di costruire argumentation framework da conversazioni estratte dai social media ed un sistema prolog in grado di generare estensioni - Arguer - ed ordinarle per importanza - Argonaut, secondo delle metriche.

Il mining proposto è stato implementato per tre diversi social media, ma le fonti da cui attingere potrebbero essere molte altre, a suggerire ampi margini di miglioramento e di estensione. Così come la creazione e di nuove metriche per la valutazione secondo logiche derivate dal contesto applicativo delle estensioni.

Tale aspetto può rappresentare uno spunto per investigare sulla possibilità di utilizzare tecniche più raffinate ed allo stato dell'arte per migliorare i vari passaggi proposti nelle due macro-aree.






% \chapter{Critical analysis}
% \input{Capitoli/Conclusioni}


\bibliographystyle{plain}
\bibliography{Bibliografia}

\end{document}
