% !TEX encoding = UTF-8
% !TEX TS-program = pdflatex
% !TEX root = ../Tesi.tex

%************************************************

%************************************************


L'argomentazione rappresenta un approccio al ragionamento nei casi in cui si dispone di conoscenza inconsistente, e può essere considerata come un metodo per gestire l’incertezza. Infatti, l’idea alla base dell'argomentazione è quella di valutare il motivo per cui un fatto sia considerato vero analizzando gli argomenti e le relazioni che intercorrono tra essi per valutarne la certezza. Tale processo può essere visto come una forma di ragionamento riguardo gli argomenti per determinarne i più accettabili. Sebbene il termine argomentare possa intuitivamente richiamare diversi significati come quello del ragionamento a partire da premesse fino alle conclusioni o l'esprimere la propria opinione in una discussione, un argomento non si lega a particolari strutture ma in senso astratto è qualsiasi cosa che può attaccare o essere attaccata da un altro argomento. Per tale motivo, un Argumentation Framework può essere adeguato a rappresentare diverse situazioni. La possibilità dell'Argumentation Framework di poter rappresentare diverse situazioni ha portato, nel tempo, alla proposta di estensioni che ponessero attenzione su diversi aspetti, come i tipi di relazione che possono sussistere o la quantificazione della forza di una relazione tra due argomenti.

\section{Argumetation Framework}
La definizione iniziale formulata da Dung prevede esclusivamente la presenza di relazioni di attacco tra argomenti e non permette di assegnare un peso a ciascuna relazione per indicare la forza di un attacco \cite{dung1995acceptability}. Successivamente, tali aspetti sono stati considerati in particolari estensioni dell'AF, come quella del Bipolar Argumentation Framework (BAF), che prevede relazioni di attacco e di supporto tra gli argomenti, e quella del Weighted Argumentation Framework (WAF), in cui le relazioni di attacco sono pesate.


\subsection{Argumetation Framework di Dung}
Si riporta di seguito la descrizione formale dell'argumentation framework di Dung \cite{dung1995acceptability}, che costituisce l’argumentation framework più semplice, contenente solamente relazioni di attacco tra argomenti.

\bigskip
\begin{defn} \textbf{Argumentation Framework (AF)}. Un Argumentation Framework (AF) è una coppia ⟨A,R⟩, dove A è un insieme di argomenti, R ⊆ A × A è una relazione binaria. Dati due argomenti a, b ∈ A, se esiste una coppia ⟨a, b⟩ ∈ R significa che a attacca b. Un insieme S ⊆ A attacca b se b è attaccato da un argomento in S. Un insieme S di argomenti attacca un insieme S' di argomenti se esiste un argomento a ∈ S che attacca un argomento b ∈ S'.
\end{defn}

Un argumentation framework $\mathcal{AF = ⟨A,R⟩}$ può essere rappresentato come un grafo orientato $\mathcal{G = ⟨V, E⟩}$, dove l’insieme dei nodi $\mathcal{V}$ corrisponde all'insieme $\mathcal{A}$ e l’insieme degli archi $\mathcal{E}$ corrisponde all'insieme $\mathcal{R}$.

\bigskip
\begin{exmp}
    Sia dato l'$\mathcal{AF = ⟨A, R⟩}$, dove:
    \begin{center}
        $\mathcal{A} = ⟨a, b, c, d, e, f, g, h⟩$
        
        $\mathcal{R} = \{⟨a, b⟩, ⟨b, c⟩, ⟨c, d⟩, ⟨a, e⟩, ⟨e, d⟩, ⟨e, f ⟩, ⟨f, g⟩, ⟨g, e⟩, ⟨f, h⟩, ⟨h, f ⟩\}$
    \end{center}
    
    \begin{figure}
      \includegraphics[width=\linewidth]{Immagini/example-graph.png}
      \caption{Rappresentazione del grafo.}
      \label{fig:graph1}
    \end{figure}
    
    \label{exm:af}
\end{exmp}

\bigskip
\begin{defn} \textbf{Difesa} Sia dato un argumentation framework $\mathcal{AF = ⟨A, R⟩}$ Un argomento a $\mathcal{∈ A}$ è difeso da un insieme di argomenti $\mathcal{S ⊆ A} $ e per ogni argomento b  $\mathcal{∈ A}$ vale che se ⟨b, a⟩ $\mathcal{∈ R}$ allora b è attaccato da $\mathcal{S}$.
\end{defn}

Ad esempio, nell'$\mathcal{AF}$ definito nell'esempio \ref{exm:af}, {a} difende c perché b, attaccante di c, è a sua volta attaccato da a.

\bigskip
\begin{defn} \textbf{Conflict-free} Sia dato un argumentation framework AF =
⟨A, R⟩. Un insieme S ⊆ A è conflict-free se e solo se non esistono due argomenti a, b ∈ S tali che ⟨a, b⟩ ∈ R.
\end{defn}  

Ad esempio, nell'$\mathcal{AF}$ definito nell'esempio \ref{exm:af}, \{a, c, g, h\} è conflict-free perché non presenta argomenti in relazione tra di loro.

\bigskip
\begin{defn} \textbf{Stable extension} Sia dato un argumentation framework
$\mathcal{AF = ⟨A, R⟩}$. Un insieme $\mathcal{S ⊆ A}$ conflict-free è una stable extension se e solo se per ogni argomento c $\mathcal{∈ A}$ tale che c $\mathcal{∉ S}$ vale che esiste un argomento b $\mathcal{∈ S}$ tale che ⟨b, c⟩ $\mathcal{∈ R}$.
\end{defn}

Ad esempio, nell'$\mathcal{AF}$ definito nell'esempio \ref{exm:af}, l'insieme delle stable extension è costituito da $\{\{a, c, f \}, \{a, c, g, h\}\}$.

\bigskip
\begin{defn} \textbf{Admissible extension} Sia dato un argumentation framework $\mathcal{AF = ⟨A, R⟩}$. Un insieme $\mathcal{S ⊆ A}$ conflict-free è una admissible extension se e solo se ogni argomento in $\mathcal{S}$ è difeso da $\mathcal{S}$.
\label{adm-ext-dung}
\end{defn}

Ad esempio, nell'$\mathcal{AF}$ definito nell'esempio \ref{exm:af}, l'insieme delle stable extension è costituito da \{∅, \{a\}, \{h\}, \{a, c\}, \{a, f\}, \{a, h\}, \{g, h\}, \{a, c, f \}, \{a, c, h\}, \{a, g, h\},
\{a, c, g, h\}\}.

\bigskip
\begin{defn} \textbf{Preferred extension} Sia dato un argumentation framework $\mathcal{AF = ⟨A, R⟩}$. Un insieme $\mathcal{S ⊆ A}$ è una preferred extension se è una admissible
extension massimale rispetto all'inclusione insiemistica.
\end{defn}

Ad esempio, nell'$\mathcal{AF}$ definito nell'esempio \ref{exm:af}, l'insieme delle preferred extension è costituito da \{\{a, c, f \}, \{a, c, g, h\}\}.

\bigskip
\begin{defn} \textbf{Complete extension} Sia dato un argumentation framework $\mathcal{AF = ⟨A, R⟩}$. Un insieme $\mathcal{S ⊆ A}$ ammissibile è una complete extension se e solo se ogni argomento che è difeso da $\mathcal{S}$ appartiene ad $\mathcal{S}$.
\end{defn}

Ad esempio, nell'$\mathcal{AF}$ definito nell'esempio \ref{exm:af}, l'insieme delle complete extension è costituito da \{\{a, c\}, \{a, c, f \}, \{a, c, g, h\}\}.

\bigskip
\begin{defn} \textbf{Grounded extension} Sia dato un argumentation framework $\mathcal{AF = ⟨A, R⟩}$. Un insieme $\mathcal{S ⊆ A}$ è la grounded extension se è una complete extension minimale rispetto all'inclusione insiemistica.
\end{defn}

Ad esempio, nell'$\mathcal{AF}$ definito nell'esempio \ref{exm:af}, la grounded extension è costituita da \{\{a, c\}\}.

\bigskip
\begin{prp}Ogni stable extension è anche una preferred extension, ma non il contrario. Inoltre, ogni preferred extension è anche una complete extension. Possono esistere più stable, preferred e complete extension, ma la grounded extension è unica.
\end{prp}

Per quanto riguarda i cicli presenti in un argumentation framework, valgono le seguenti proprietà:

\bigskip
\begin{prp}. Se un argumentation framework $\mathcal{AF = ⟨A,R⟩}$ non ha alcun ciclo di lunghezza pari, allora l’unica preferred extension di AF è l’insieme vuoto.
\end{prp}

\bigskip
\begin{prp}. Se un argumentation framework $\mathcal{AF = ⟨A,R⟩}$ non ha alcun ciclo di lunghezza dispari, allora ogni preferred extension di AF è una stable extension.
\end{prp}

\bigskip
\begin{prp}. Se un argumentation framework $\mathcal{AF = ⟨A,R⟩}$ non ha alcun ciclo ed $\mathcal{A = ∅}$, allora $\mathcal{AF}$ ha una singola estensione che è preferred, completa e grounded.
\end{prp}

\begin{figure}
    \includegraphics[width=\linewidth]{Immagini/semantics-relationships.png}
    \caption{Relazioni tra le semantiche.}
    \label{fig:baf-graph1}
\end{figure}


\subsection{Bipolar Argumetation Framework}
Una estensione dell'argumentation framework di Dung considera il fatto che tra due argomenti non esiste solo una relazione di attacco, ma può esistere anche una relazione di supporto. La distinzione in due tipi di relazioni opposte tra di loro suggerisce la nozione di bipolarismo, ovvero l’esistenza di due tipi di informazioni indipendenti che hanno natura diametralmente opposta e che rappresentano forze di repulsione. Il concetto di bipolarismo è importante quando si devono prendere decisioni in quanto è possibile distinguere informazioni a supporto di una decisione, ed informazioni che la contrastano [1]. Si descrive, di seguito, formalmente il bipolar argumentation framework.

\bigskip
\begin{defn} \textbf{Bipolar Argumentation Framework (BAF)} Un Bipolar Argumentation Framework (BAF) è una tripla $⟨\mathcal{A}, \mathcal{R}_{def}, \mathcal{R}_{sup}⟩$ dove $\mathcal{A}$ è un insieme di argomenti, $\mathcal{R}_{def}$ è una relazione binaria su $\mathcal{A}$ chiamata relazione di attacco e $\mathcal{R}_{sup}$ è una relazione binaria su $\mathcal{A}$ chiamata relazione di supporto. Dati due argomenti $a, b ∈ \mathcal{A}$, $a \mathcal{R}_{def} b ⇔ ⟨a, b⟩ ∈ \mathcal{R}_{def}$ (rispettivamente  $a \mathcal{R}_{sup} b ⇔ ⟨a, b⟩ ∈  \mathcal{R}_{sup}$ ) indica che a attacca b (rispettivamente a supporta b). Un bipolar argumentation framework $BAF = ⟨\mathcal{A}, \mathcal{R}_{def} , \mathcal{R}_{sup} ⟩$ può essere rappresentato come un grafo orientato $\mathcal{G} = ⟨\mathcal{V}, \mathcal{E}⟩$, dove l'insieme dei nodi $\mathcal{V}$ corrisponde all'insieme $\mathcal{A}$ e l'insieme degli archi $\mathcal{E}$ corrisponde all'insieme $\mathcal{R}_{def} ∪ \mathcal{R}_{sup}$. Dati due argomenti $a, b ∈ \mathcal{A}$, $a \mathcal{R}_{def} b$ è rappresentato come $a → b$ e $a \mathcal{R}_{sup} b$ è rappresentato come $a -→ b$.

\end{defn}

\bigskip
\begin{exmp}
    Sia dato il $BAF  = ⟨\mathcal{A}, \mathcal{R}_{def}, \mathcal{R}_{sup}⟩$, dove:
    \begin{center}
        $\mathcal{A} = ⟨a, b, c, d, e, f, g, h⟩$
        
        $\mathcal{R}_{def} = \{⟨c, d⟩, ⟨a, e⟩, ⟨e, f⟩, ⟨f, g⟩, ⟨g, e⟩, ⟨f, h⟩, ⟨h, f⟩\}$
        
        $\mathcal{R}_{sup} = \{⟨a, b⟩, ⟨b, c⟩, ⟨e, d⟩\}$
    \end{center}
    
    \begin{figure}
      \includegraphics[width=\linewidth]{Immagini/example-baf-graph.png}
      \caption{Rappresentazione del grafo bipolar.}
      \label{fig:baf-graph1}
    \end{figure}
    
    \label{exm:baf}
\end{exmp}

\bigskip
\begin{defn} \textbf{Relazione di Attacco Diretto} 
Sia dato un bipolar argumentation framework $BAF = ⟨\mathcal{A}, \mathcal{R}_{def}, \mathcal{R}_{sup}⟩$ e siano dati due argomenti $a, b ∈ \mathcal{A}$. Se $ a \mathcal{R}_{def} b$ allora si parla di attacco diretto da a verso b.
\label{defn:adbaf}
\end{defn}

Ad esempio, nel $BAF$ definito nell'esempio \ref{exm:baf}, è presente una relazione di attacco diretto da $c$ verso $d$ perchè $⟨c, d⟩ ∈ \mathcal{R}_{def}$.


\bigskip
\begin{defn} \textbf{Relazione di Attacco Indiretto} 
Sia dato un bipolar argumentation framework $BAF = ⟨\mathcal{A}, \mathcal{R}_{def}, \mathcal{R}_{sup}⟩$. Un attacco indiretto verso un argomento $b ∈ \mathcal{A}$ è una sequenza $a_{1}\mathcal{R}_{1}...\mathcal{R}_{n-1}a_{n}$ dove $n \geq 3, a_n = b, \forall i = 2, ..., n-1, \mathcal{R}_i = \mathcal{R}_{sup}$ e $ \mathcal{R}_{1} = \mathcal{R}_{def}$.
\label{defn:aibaf}
\end{defn}

Ad esempio, nel $BAF$ definito nell'esempio \ref{exm:baf}, è presente una relazione di attacco indiretto verso $d$ perchè $⟨a, e⟩ ∈ \mathcal{R}_{def}$ e $⟨e, d⟩ ∈ \mathcal{R}_{sup}$.


\bigskip
\begin{defn} \textbf{Attacco supportato} 
Sia dato un bipolar argumentation framework $BAF = ⟨\mathcal{A}, \mathcal{R}_{def}, \mathcal{R}_{sup}⟩$. Un attacco supportato verso un argomento $b ∈ \mathcal{A}$ è una sequenza $a_{1}\mathcal{R}_{1}...\mathcal{R}_{n-1}a_{n}$ dove $n \geq 3, a_n = b, \forall i = 1, ..., n-2, \mathcal{R}_i = \mathcal{R}_{sup}$ e $ \mathcal{R}_{n-1} = \mathcal{R}_{def}$.
\label{defn:asbaf}
\end{defn}

Ad esempio, nel $BAF$ definito nell'esempio \ref{exm:baf}, è presente un attacco supportato verso $d$ perchè $⟨c, d⟩ ∈ \mathcal{R}_{def}$ e $⟨a, b⟩, ⟨b, c⟩ ∈ \mathcal{R}_{sup}$.


\bigskip
\begin{defn} \textbf{Attacco di insieme} 
Sia dato un bipolar argumentation framework $BAF = ⟨\mathcal{A}, \mathcal{R}_{def}, \mathcal{R}_{sup}⟩$. Un insieme $S \subseteq A$ attacca un argomento $b \in A$ se e solo se esiste un attacco diretto, supportato o indiretto verso $b$ da un elemento di $S$.
\label{defn:adibaf}
\end{defn}

Ad esempio, nel $BAF$ definito nell'esempio \ref{exm:baf}, è presente un attacco di insieme da $S = =\{a, b, c\}$ verso $d$ perchè esiste un attacco supportato dagli elementi $a, b, c \in S$ verso $d$.


\bigskip
\begin{defn} \textbf{Supporto di insieme} 
Sia dato un bipolar argumentation framework $BAF = ⟨\mathcal{A}, \mathcal{R}_{def}, \mathcal{R}_{sup}⟩$. Un insieme $S \subseteq A$ supporta un argomento $b \in A$ se e solo se esiste una sequenza $a_{1}\mathcal{R}_{sup}...\mathcal{R}_{sup}a_{n}$ dove $n \geq 2, a_n = b$ e $a_1 \in S$
\label{defn:sdibaf}
\end{defn}

Ad esempio, nel $BAF$ definito nell'esempio \ref{exm:baf}, è presente un supporto di insieme da $S = =\{a\}$ verso $c$ perchè esiste una sequenza di supporti da $a \in S$ verso $c$.


\bigskip
\begin{defn} \textbf{Difesa} 
Sia dato un bipolar argumentation framework $BAF = ⟨\mathcal{A}, \mathcal{R}_{def}, \mathcal{R}_{sup}⟩$. Un insieme $S \subseteq A$ difende un argomento $a \in A$ se e solo se per ogni argomento $b \in A$, se $b$ attacca $a$ allora $b$ è attaccato da $S$.
\label{defn:dbaf}
\end{defn}

Ad esempio, nel $BAF$ definito nell'esempio \ref{exm:baf}, $e$ difende $g$ perchè $f$, attaccante di $g$, è a sua volta attaccato da $e$.


\bigskip
\begin{defn} \textbf{Conflict-free} 
Sia dato un bipolar argumentation framework $BAF = ⟨\mathcal{A}, \mathcal{R}_{def}, \mathcal{R}_{sup}⟩$. Un insieme $S \subseteq A$ è conflict-free se e solo se non esistono due argomenti $a, b \in S$ tali che sussista un attacco di insieme da $\{a\}$ verso $b$.
\end{defn}

Ad esempio, nel $BAF$ definito nell'esempio \ref{exm:baf}, $\{a, b, c, f\}$ è conflict-free perchè non presenta argomenti coinvolti in attacchi di insieme tra loro.


\bigskip
\begin{defn} \textbf{Safe} 
Sia dato un bipolar argumentation framework $BAF = ⟨\mathcal{A}, \mathcal{R}_{def}, \mathcal{R}_{sup}⟩$. Un insieme $S \subseteq A$ è safe se e solo se non esiste alcun argomento $b \in A$ tale che sussista un attacco di insieme da $\{S\}$ verso $b$ o che ci sia un supporto di insieme da $S$ verso $b$ o che $b \in S$. 
\end{defn}

Ad esempio, nel $BAF$ definito nell'esempio \ref{exm:baf}, $\{b, c\}$ è safe perchè attacca solo $d$, ma non lo supporta; inoltre $d$ non appartiene a tale insieme.


\bigskip
\begin{defn} \textbf{Stable Extension} 
Sia dato un bipolar argumentation framework $BAF = ⟨\mathcal{A}, \mathcal{R}_{def}, \mathcal{R}_{sup}⟩$. Un insieme $S \subseteq A$ conflict-free è una stable extension se e solo se per ogni argomento $a \notin S$ sussiste un attacco di insieme da $S$ verso $a$.
\end{defn}

Ad esempio, nel $BAF$ definito nell'esempio \ref{exm:baf}, l'insieme delle stable extension è costituito da $\{\{a, b, c, f\}, \{a, b, c, g, h\}\}$.


\bigskip
\begin{defn} \textbf{d-admissible extension} 
Sia dato un bipolar argumentation framework $BAF = ⟨\mathcal{A}, \mathcal{R}_{def}, \mathcal{R}_{sup}⟩$. Un insieme $S \subseteq A$ conflict-free è una d-admissible extension se e solo se ogni argomento in $S$ è difeso da $S$. Tale definizione equivale alla definizione \ref{adm-ext-dung} di ammissibilità per l'AF di Dung.
\end{defn}

Ad esempio, nel $BAF$ definito nell'esempio \ref{exm:baf}, l'insieme delle d-admissible extension è costituito da $\{∅, \{a, h\}$, $\{a, b, h\}$, $\{a, c, h\}$, $\{a, b, c, h\}$, $\{a, g, h\}$, $\{a, b, g, h\}$, $\{a, c, g, h\}$, $\{a, b, c, g, h\}$, $\{g, h\}$, $\{c, g, h\}$, $\{b, g, h\}$, $\{b, c, g, h\}$, $\{b, h\}$, $\{b, c, h\}$, $\{c, h\}$, $\{h\}$, $\{a, f\}$, $\{a, b, f \}$, $\{a, c, f \}$, $\{a, b, c, f \}$, $\{a\}$, $\{a, b\}$, $\{a, c\}$, $\{a, b, c\}$, $\{b\}$, $\{b, c\}$, $\{c\}\}$.


\bigskip
\begin{defn} \textbf{s-admissible extension} 
Sia dato un bipolar argumentation framework $BAF = ⟨\mathcal{A}, \mathcal{R}_{def}, \mathcal{R}_{sup}⟩$. Un insieme $S \subseteq A$ safe è una s-admissible extension se e solo se ogni argomento in $S$ è difeso da $S$.
\end{defn}

Ad esempio, nel $BAF$ definito nell'esempio \ref{exm:baf}, l'insieme delle s-admissible extension è costituito da $\{∅, \{a, h\}$, $\{a, b, h\}$, $\{a, c, h\}$, $\{a, b, c, h\}$, $\{a, g, h\}$, $\{a, b, g, h\}$, $\{a, c, g, h\}$, $\{a, b, c, g, h\}$, $\{g, h\}$, $\{c, g, h\}$, $\{b, g, h\}$, $\{b, c, g, h\}$, $\{b, h\}$, $\{b, c, h\}$, $\{c, h\}$, $\{h\}$, $\{a, f \}$, $\{a, b, f \}$, $\{a, c, f \}$, $\{a, b, c, f \}$, $\{a\}$, $\{a, b\}$, $\{a, c\}$, $\{a, b, c\}$, $\{b\}$, $\{b, c\}$, $\{c\}\}$.


\bigskip
\begin{defn} \textbf{c-admissible extension}
Sia dato un bipolar argumentation framework $BAF = ⟨\mathcal{A}, \mathcal{R}_{def}, \mathcal{R}_{sup}⟩$. Un insieme $S \subseteq A$ conflict-free è una c-admissible extension se e solo se $S$ è chiuso rispetto ad $\mathcal{R}_{sup}$ ed ogni argomento in $S$ è difeso da $S$.
\end{defn}

Ad esempio, nel $BAF$ definito nell'esempio \ref{exm:baf}, l'insieme delle c-admissible extension è costituito da $\{∅, \{a, b, c, f \}$, $\{a, b, c, h\}$, $\{a, b, c, g, h\}$, $\{a, b, c\}$, $\{g, h\}$, $\{h\}\}$.


\bigskip
\begin{defn} \textbf{d-preferred (risp. s-preferred, c-preferred) extension}
Sia dato un bipolar argumentation framework $BAF = ⟨\mathcal{A}, \mathcal{R}_{def}, \mathcal{R}_{sup}⟩$. Un insieme $S \subseteq A$ è una d-preferred (risp. s-preferred, c-preferred) extension se è una d-admissible (risp. s-admissible, c-admissible) extension massimale rispetto all'inclusione insiemistica.
\end{defn}

Ad esempio, nel $BAF$ definito nell'esempio \ref{exm:baf}, l'insieme delle d-preferred, s-preferred, c-preferred extension sono tutti costituiti da $\{\{a,b,c,g,h\}$, \\ $\{a,b,c,f\}\}$.



\begin{table}[H]
    \centering
    \begin{tabular}{|l|c|c|}
        \hline
        \textbf{Extension} & \textbf{\textit{AF}} & \textit{\textbf{BAF}} \\ \hline
        \textbf{Stable}       & x & x \\ \hline
        \textbf{d-admissible} & x & x \\ \hline
        \textbf{s-admissible} &   & x \\ \hline
        \textbf{c-admissible} &   & x \\ \hline
        \textbf{Complete}     & x &   \\ \hline
        \textbf{Grounded}     & x &   \\ \hline
        \textbf{d-preferred}  & x & x \\ \hline
        \textbf{s-preferred}  &   & x \\ \hline
        \textbf{c-preferred}  &   & x \\ \hline
    \end{tabular}
    \caption{Tabella di riepilogo delle estensioni calcolabili per \textit{AF} e \textit{BAF}.}
\end{table}


\subsection{Weighted Argumetation Framework}
Una ulteriore proposta che segue una direzione differente è l'estensione dell'argumentation framework di Dung in cui gli attacchi tra argomenti sono associati ad un peso numerico che indica la forza di un attacco. Ad esempio, si considerino i seguenti argomenti:

\begin{itemize}
    \item \textbf{a1} "The house is in a good location, it is large enough for our family and it is affordable: we should buy it".
    \item \textbf{a2} "The house suffers from subsidence, which would be prohibitively expensive to fix: we should not buy it".
\end{itemize}

Entrambi gli argomenti si attaccano a vicenda poiché in contrasto tra di loro, per cui nel caso dell'argumentation framework di Dung non sarà possibile individuare una estensione grounded. Infatti, tale rappresentazione non tiene conto del fatto che gli attacchi non hanno uguale peso. Nell'esempio riportato, \textbf{a2} attacca in modo più forte \textbf{a1}, per cui tale attacco assume un peso maggiore. Considerando tali pesi, viene introdotto un parametro $\beta$, denominato inconsistency budget, con l'obiettivo di filtrare le relazioni ed ottenere una soluzione diversa, in cui determinati argomenti possono essere considerati accettabili rispetto ad una soglia di tolleranza dell'inconsistenza \cite{dunne2011weighted}. Di seguito si descrivono formalmente il Weighted Argumentation Framework e l'inconsistency budget.


\begin{defn} \textbf{Weighted Argumentation Framework (WAF)}
Un Weighted Argumentation Framework $(WAF)$ è una tripla $⟨\mathcal{A}, \mathcal{R}, w⟩$, dove $⟨\mathcal{A}, \mathcal{R}⟩$ è un Argumentation Framework di Dung e $w: \mathcal{R} \rightarrow \mathbb{R}^{+}$ è una funzione che assegna numeri reali strettamente positivi come peso agli attacchi.
\end{defn}

Un Weighted Argumentation Framework $WAF = ⟨\mathcal{A}, \mathcal{R}, w⟩$ può essere rappresentato come un grafo orientato e pesato $\mathcal{G} = ⟨\mathcal{V}, \mathcal{E}⟩$, dove l'insieme dei nodi $\mathcal{V}$ corrisponde all'insieme $\mathcal{A}$ e l'insieme degli archi $\mathcal{E}$ corrisponde all'insieme $\mathcal{R}$.

\begin{exmp}
    Sia dato il $WAF  = ⟨\mathcal{A}, \mathcal{R}, w⟩$, dove:
    \begin{center}
        $\mathcal{A} = ⟨a, b, c, d, e, f, g, h⟩$
        
        $\mathcal{R} = \{⟨a, b⟩, ⟨b, c⟩, ⟨c, d⟩, ⟨a, e⟩, ⟨e, d⟩, ⟨e, f⟩, ⟨f, g⟩, ⟨g, e⟩, ⟨f, h⟩, ⟨h, f⟩\}$
        
        $w(⟨a, b⟩) = 2, w(⟨b, c⟩) = 1, w(⟨c, d⟩) = 3, w(⟨a, e⟩) = 5, w(⟨e, d⟩) = 7, w(⟨e, f⟩) = 2, w(⟨f, g⟩) = 2, w(⟨g, e⟩) = 3, w(⟨f, h⟩) = 5, w(⟨h, f⟩) = 7$
    \end{center}
    
    \begin{figure}[h]
      \includegraphics[width=\linewidth]{Immagini/example-waf-graph.png}
      \caption{Rappresentazione del grafo weighted.}
      \label{fig:waf-graph1}
    \end{figure}
    
    \label{exm:waf}
\end{exmp}

Ad esempi nel $WAF$ definito nell'esempio \ref{exm:waf}, l'insieme delle admissible extension è costituito da \{∅, \{a, c, f \}$, $\{a, f \}$, $\{a, c\}$, $\{a\}$, $\{g, h\}$, $\{h\}$, $\{a, c, g, h\}$, $\{a, c, h\}$, $\{a, g, h\}$, $\{a, h\}\}. L'insieme delle preferred extension è costituito da $\{\{a, c, f\}$, $\{a, c, g, h\}\}$. L'insieme delle complete extension è costituito da $\{\{a, c, g, h\}$, $\{a, c\}, \{a, c, f\}\}$. La grounded extension è costituita da $\{\{a, c\}\}$.

% ===================================================================
% ===================================================================
% ===================================================================
% ===================================================================

\subsection{Bipolar Weighted Argumetation Framework}

\begin{defn} \textbf{Bipolar Weighted Argumentation Framework}
Un Bipolar Weighted Argumentation Framework $(BWAF)$ è una tripla $\mathcal{A}$ è un insieme di argomenti, $\mathcal{R}$ è una relazione binaria e $w: \mathcal{R} \rightarrow [-1, 0)$ è una funzione che assegna un peso a ciascuna relazione. Dati due argomenti $a, b \in \mathcal{A}, a\mathcal{R}b \iff ⟨a,b ⟩\in \mathcal{R}$ indica che ha $a$ attacca $b$ se $-1 \leq (⟨a,b⟩) < 0$, altrimenti che $a$ supporta $b$ se $0 < w(⟨a,b⟩) \leq 1$.
\end{defn}

\bigskip
Un Bipolar Weighted Argumentation Framework $BWAF = ⟨\mathcal{A}, \mathcal{R}, w⟩$ può essere rappresentato come un grafo orientato e pesato $\mathcal{G} = ⟨\mathcal{V}, \mathcal{E}⟩$, dove l'insieme dei nodi $\mathcal{V}$ corrisponde all'insieme $\mathcal{A}$ e l'insieme degli archi $\mathcal{E}$ corrisponde all'insieme $\mathcal{R}$.

\bigskip
\begin{defn} \textbf{Relazione di attacco e di supporto}
Sia dato un Bipolar Weighted Argumentation Framework $BWAF = ⟨\mathcal{A}, \mathcal{R}, w⟩$. Definiamo la relazione binaria di attacco tra gli elementi di $\mathcal{A}$ come $\mathcal{R}_{def} = \{⟨a,b⟩ \in \mathcal{R} \ | \ -1 \leq w(⟨a, b⟩) < 0\} \subseteq \mathcal{R}$ e la relazione binaria di supporto tra gli elementi di $\mathcal{A}$ come $\mathcal{R}_{sup} = \{⟨a, b⟩ \in \mathcal{R} \ | \ 0 < w(⟨a, b⟩) \leq 1\} \subseteq \mathcal{R}$.
\end{defn}

\bigskip
\begin{exmp}
    Sia dato il $BWAF  = ⟨\mathcal{A}, \mathcal{R}, w⟩$, dove:
    \begin{center}
        $\mathcal{A} = ⟨a, b, c, d, e, f, g, h⟩$
        
        $\mathcal{R}_{def} = \{⟨c, d⟩, ⟨a, e⟩, ⟨e, f⟩, ⟨f, g⟩, ⟨g, e⟩, ⟨f, h⟩, ⟨h, f⟩\}$
        
        $\mathcal{R}_{sup} = \{⟨a, b⟩, ⟨b, c⟩, ⟨e, d⟩\}$
        
        $w(⟨a, b⟩) = 0.7, w(⟨b, c⟩) = 0.9, w(⟨c, d⟩) = -0.4, w(⟨a, e⟩) = -0.7, w(⟨e, d⟩) = 0.3, w(⟨e, f⟩) = -0.5, w(⟨f, g⟩) = -0.3, w(⟨g, e⟩) = -0.1, w(⟨f, h⟩) = -0.5, w(⟨h, f⟩) = -0.7$
    \end{center}
    
    \begin{figure}[h]
      \includegraphics[width=\linewidth]{Immagini/example-bwaf-graph.png}
      \caption{Rappresentazione del BWAF.}
      \label{fig:bwaf-graph1}
    \end{figure}
    
    \label{exm:bwaf}
\end{exmp}

Alla luce delle definizioni di relazione di attacco $\mathcal{R}_{def}$ e di relazione di supporto $\mathcal{R}_{sup}$ riportate per il Bipolar Weighted Framework $BWAF = ⟨\mathcal{A}, \mathcal{R}, w⟩$, valgono le definizioni \ref{defn:adbaf}, \ref{defn:aibaf}, \ref{defn:asbaf}, \ref{defn:adibaf}, \ref{defn:sdibaf}, \ref{defn:dbaf} espresse per il $BAF = ⟨\mathcal{A}, \mathcal{R}_{def}, \mathcal{R}_{sup}⟩$.

\bigskip
\begin{defn} \textbf{Inconsistency budget}
Siano dati un bipolar argumentation $BWAF = ⟨\mathcal{A}, \mathcal{R}, w⟩$, una relazione $\mathcal{X} \subseteq \mathcal{R}$ e due inconsistency budget $\alpha \in \mathbb{R}_{<0}$, $\beta \in \mathbb{R}_{>0}$. Si definiscono le funzioni:

    \begin{center}
        
        $$wt(\mathcal{X}, w) = \sum_{⟨a, b⟩ \in \mathcal{X}}$$
        
        $sub_{def}(\mathcal{R},w,\alpha) = \{Y \ | \ Y \subseteq \mathcal{R} \land wt(\mathcal{R},w) \geq \alpha \}$
        
        $sub_{sup}(\mathcal{R},w,\beta) = \{Y \ | \ Y \subseteq \mathcal{R} \land wt(\mathcal{R},w) \leq \alpha \}$
        
        $\phi(\mathcal{A}, \mathcal{R}_{def}, \mathcal{R}_{sup}, \alpha, \beta) = 
        \{\mathcal{S} \subseteq \mathcal{A} \ | \ \exists \mathcal{R}_{\sigma} \in \{sub_{def}(\mathcal{R}_{def}, w, \alpha) \ \lor \ sub_{sup}(\mathcal{R}_{sup}, w, \beta) \} \land S \in \Theta (\mathcal{A}, \mathcal{R} \setminus \mathcal{R}_{\sigma}) \}$
        
    \end{center}
    
    dove $\Theta(\mathcal{A}, \mathcal{B})$ è un operatore che restituisce gli argomenti presenti in  $\mathcal{A}$ coinvolti nella relazione $\mathcal{B}$.

\end{defn}

A partire da $\phi(\mathcal{A}, \mathcal{R}_{def}, \mathcal{R}_{sup}, \alpha, \beta)$ si ottiene un sottoinsieme dell'insieme delle parti di $\mathcal{A}$ i cui elementi contengono solo argomenti che partecipano a relazioni con un peso inferiore ad $\alpha$ e superiore a $\beta$. Un insieme $\mathcal{S} \in \phi(\mathcal{A}, \mathcal{R}, \beta)$ può rientrare nelle estensioni definite per il framework di Dung al variare del parametro $\beta$. Per tale motivo, nei weighted argumentation framework, si parla di estensioni come la $\beta$-admissible, $\beta$-preferred e la $\beta$-grounded.
Ad esempio, dati gli inconsistency budget $\alpha = -0.3$ e $\beta = 0.3$, nel $BWAF$ definito nell'esempio \ref{exm:bwaf}, l'insieme delle $\alpha$-$\beta$-admissible extension è costituito da:  

\begin{center}
    $\{\varnothing,\{a,b,c,f,g\},\{a,c,f,g\},\{a,b,f,g\},\{a,f,g\},\{a,b,c,f\},\{a,c,f\},$ \\ 
    $\{a,b,f\},\{a,f\},\{a,b,c,g,h\},\{a,c,g,h\},\{a,b,g,h\},\{a,g,h\},\{a,b,c,h\},$ \\ 
    $\{a,c,h\},\{a,b,h\},\{a,h\},\{b,c,g,h\},\{b,g,h\},\{b,c,h\},\{b,h\},\{c,g,h\},$ \\ 
    $\{c,h\},\{g,h\},\{h\},\{a,b,c,g\},\{a,c,g\},\{a,b,g\},\{a,g\},\{a, b, c\}, \{a, c\}, \{a, b\},$ \\
    $\{a\}, \{b, c, g\}, \{b, g\},\{b, c\}, \{b\}, \{c, g\}, \{c\}, \{g\}\}$
\end{center}

Tale insieme è safe, per cui esso costituisce anche la $\alpha$-$\beta$-c-admissible extension del $BWAF$. L'insieme delle $\alpha$-$\beta$-c-admissible extension è, invece:

\begin{center}
    $\{\{g\},\{\},\{g,h\},\{h\},\{a,b,c,f,g\},\{a,b,c,f\},$ \\ 
    $\{a,b,c,g\},\{a,b,c\},\{a,b,c,g,h\},\{a,b,c,h\}$
\end{center}

Inoltre, l'insieme delle $\alpha$-$\beta$-d-preferred extension è costituito da:

\begin{center}
    $\{a,b,c,g,h\},\{a,b,c,f,g\}$
\end{center}

Tale insieme è safe, per cui esso costituisce anche la $\alpha$-$\beta$-s-preferred extension del $BWAF$. Inoltre, tale insieme soddisfa la proprietà di chiusura rispetto alla relazione di supporto e tale insieme difende se stesso, per cui è anche una $\alpha$-$\beta$-c-preferred extension.
Nella formalizzazione del $BWAF$ introdotta in questa sezione, si aggiunge un operatore in grado di valutare la forza di un attacco o di un supporto da un argomento verso un argomento non necessariamente in relazione diretta con esso. Si introduce, di seguito, la formalizzazione di tale operatore.

\bigskip
\begin{defn} \textbf{Argue}
Sia dato un Bipolar Weighted Argumentation Framework $BWAF = ⟨\mathcal{A}, \mathcal{R}, w⟩$. Definiamo la funzione$ path\_weight$ che calcola la forza di un cammino $v_1, ... , v_n$, dove $v_1 = a$, $v_n = b$ come il prodotto di $w(⟨v_{i-1},v_1⟩)$, $\forall i = 2,...,n$. Definiamo la funzione $node\_influence$ che calcola l'influenza di un argomento $a \in \mathcal{A}$ in base ai cicli a cui appartiene come $\prod_{a,...,a}$  $path\_weight(⟨a,...,a⟩)$, se $a$ appartiene ad almeno un ciclo, 1 altrimenti. Definiamo la funzione argue che calcola la forza di una relazione da $a \in A$ verso $b \in A$ nel seguente modo:

    \begin{center}
        $$argue(a,b) = \sum_{a,...,b} path_weight(⟨a,...,b⟩). \prod_{c \in ⟨a,...,b⟩} node\_influence(c)$$
    \end{center}

\end{defn}

In particolare, dati due argomenti $a$ e $b$, la definizione per la $argue(a,b)$ equivale a calcolare la somma delle forze di tutti i cammini da $a$ a $b$, ciascuna delle quali è ottenuta  come il prodotto dei pesi delle relazioni contenute nel cammino moltiplicato per il prodotto dell'influenza di ciascun argomento presente nel cammino. 

La motivazione legata a tale definizione è dovuta al fatto che tra due argomenti possono sussistere
cammini multipli costituiti da relazioni di attacco e di supporto, che hanno rispettivamente un
peso negativo e positivo. Inoltre, un argomento può essere coinvolto in più cicli, ciascuno dei
quali può contenere argomenti a loro volta coinvolti in altri cicli. 

Secondo la definizione data, la $argue(a,b)$ presenta un valore positivo se sussiste una difesa da $a$ verso $b$, negative sussiste un attacco tra di essi. Tale funzione rappresenta una misura complessiva della relazione che intercorre tra due argomenti nell'ambito di una intera discussione. Nell'esempio \ref{exm:bwaf}, la relazione da $a$ verso $d$ è calcolata nel seguente modo:

\begin{center}

    $path\_weight(⟨a,b,c,d⟩) = 0.7 \cdot 0.9 \cdot (-0.4) = -0.252$
    
    $path\_weight(⟨a,e,d⟩) = (-0.7) \cdot 0.3 = -0.21$
    
    $node\_influence(e) = path\_weight(⟨e,f,h,f,g,e⟩) =$ \\ 
    $= (-0.5) \cdot (-0.5) \cdot(-0.5) \cdot (-0.7) \cdot (-0.3) \cdot (-0.1) =$ \\
    $= -0.2508975$
    
\end{center}

Dall'esempio si evince che $a$ attacca $d$ con una forza pari a $a \approx -0.25$


\section{Argumetation Mining}
 L'obiettivo dell'Argumentation Mining, un campo di ricerca in forte crescita, è la creazione di nuovi metodi per estrarre Argumentation Framework da fonti dati non strutturate, principalmente testuali, ed include task come la identificazione degli argomenti e delle relazioni tra loro. Molti ricercatori hanno già applicato l'Argumentation Mining in molti domini. Ad esempio in \cite{teufel1999annotation} si cerca di identificare argomenti nelle frasi di pubblicazioni scientifiche, in \cite{moens2007automatic} gli argomenti vengono estratti da documenti legali, in \cite{feng2011classifying} vengono analizzati articoli di giornale o in \cite{florou2013argument} nelle trascrizioni di casi giudiziari. Recentemente ha guadagnato popolarità grazie alla possibilità di elaborare informazioni provenienti dal Web, ed in particolare dai Social Media.

In generale, l'output delle analisi automatiche svolte su grandi quantità di dati provenienti da Social Media (o dal Web), può fornire dimostrazioni riguardo un certo tema, potenzialmente controverso, o aiutare a rivelare falle nelle argomentazioni.

Essendo un campo di ricerca emergente, l'Argumentation Mining soffre di una carenza di dataset etichettati, che risultano cruciali per la progettazione, l'apprendimento e la valutazione di algoritmi. 

Boltuzic  and Snajder \cite{boltuvzic2014} evidenziano \textit{"a differenza dei dibattiti o di altre fonti di argomentazione più formali, gli argomenti forniti dagli utenti, se esistenti, sono meno formali, ambigui, vaghi, impliciti o spesso semplicemente mal formulati"}. Bentahar, Moulin, e Be langer \cite{bentahar2010} propongono una tassonomia dei modelli di argomentazione che è suddivisa orizzontalmente in tre categorie: modelli a micro-level, modelli a macro-level e modelli retorici.

\begin{itemize}
    \item \textbf{Micro-level Argumentation} si concentra sulla struttura dei singoli argomenti
    \item \textbf{Macro-level Argumentation} in contrasto, si concentrano sulla struttura della discussione
    \item \textbf{Rhetorical Models} si occupa di argomenti, che sono basati sulla percezione del mondo, e la valutazione soggettiva piuttosto che stabilire la verità di una proposizione
\end{itemize}

Secondo la classica teoria aristotelica, l'argomento può esistere in tre dimensioni, che sono \textit{logos}, \textit{pathos} ed \textit{ethos}. \\
La dimensione \textit{logos} rappresenta una dimostrazione della ragione, un tentativo di convincere stabilendo un argomento logico. Ad esempio, il sillogismo appartiene a questa dimensione dell'argomentazione. La dimensione pathos fa appello alle emozioni del ricevente e influisce sulla sua cognizione. La dimensione dell'ethos dell'argomento si basa sulla credibilità dell'interlocutore.


\subsection{Approcci}
La separazione di elementi testuali argomentativi da quelli non argomentativi è il primo passo nella pipeline dell'Argumentation Mining. Una volta individuati, si estraggono gli argomenti e la loro struttura. Questo task è solitamente considerato un problema di classificazione binaria.
Uno dei primi approcci è stato proposto da  Moens et al. \cite{ashley2013}, basato sulla identificazione di elementi testuali argomentativi in giornali e documenti legali.
Un approccio simile è proposto da Florou et al. \cite{florou2013}, dove classificano documenti ricavati attraverso un crawler già predisposto all'estrazione di segmenti di testo contenenti argomenti o meno. A tale scopo, utilizzano diversi indicatori del discorso e caratteristiche estratte dal tempo e umore dei verbi. Sebbene la separazione delle unità di testo argomentative da quelle non argomentative sia un passo importante nell'argumentation mining, consente semplicemente il rilevamento di unità di testo rilevanti per l'argomentazione e non rivela il ruolo argomentativo dei componenti dell'argomento.

\subsubsection{Classificazione dei componenti dell'argomento}
La classificazione dei componenti argomentativi mira a identificare il ruolo (ad esempio conclusioni e premesse) dei componenti dell'argomento. Uno dei primi approcci per identificare i componenti dell'argomento è la Argumentative Zoning proposta da \cite{teufel1999}. Ogni frase è classificata come uno dei sette ruoli retorici tra cui ad es. reclamo, risultato o scopo usando caratteristiche strutturali, lessicali e sintattiche. L'assunto di base di questo lavoro è che le componenti argomentative estratte da un articolo scientifico forniscono un buon riassunto del suo contenuto. Rooney et al. \cite{rooney2012applying} si concentrano anche sull'identificazione delle componenti dell'argomento, ma a differenza del lavoro di Teufel \cite{teufel1999}, il loro schema non è adattato ad un particolare genere. Nei loro esperimenti, identificano affermazioni, premesse e unità di testo non argomentative nel corpus di Araucaria. Feng e Hirst \cite{feng2011classifying} usano anche il nucleo di Araucaria per i loro esperimenti, ma si concentrano sull'identificazione di schemi di argomentazione (\cite{walton2002argumentation} che sono template per argomenti. Poiché il loro approccio si basa su caratteristiche estratte da informazioni mutue di premesse e conclusioni, richiede che le componenti dell'argomento vengano identificate attendibilmente in anticipo. 
Palau and Moens \cite{palau2009argumentation} riportano diversi esperimenti per la classificazione di componenti di argomenti. Si concentrano esclusivamente sul settore legale e, in particolare, sulle cause giudiziarie della Corte europea dei diritti dell'uomo (CEDU). Considerano la classificazione dei componenti di argo come due passaggi consecutivi. Utilizzano un modello di maximum-entropy per l'identificazione di unità testuali argomentative prima di identificare il ruolo descrittivo (conclusioni e premessa) dei componenti identificati utilizzando una Support Vector Machine. 

\subsubsection{Identificazione delle strutture di argomentazione}
Attualmente, ci sono solo pochi approcci finalizzati all'identificazione delle strutture di argomentazione. Ad esempio, l'approccio proposto da Palau e Moens \cite{palau2009argumentation} si basa su una grammatica Context-Free (CFG) creata manualmente e sulla presenza di marker di discorso per identificare una struttura ad albero tra i componenti degli argomenti. Tuttavia, l'approccio si basa sulla presenza di indicatori di discorso e sfrutta le regole create manualmente. Pertanto, non contiene argomenti mal formattati e non è in grado di identificare strutture di argomentazione implicite che sono comuni nel discorso argomentativo. Un altro approccio è stato presentato da Cabrio e Villata \cite{cabrio2012combining} identificando le relazioni tra gli argomenti di una piattaforma di dibattito online per identificare gli argomenti accettati e sostenere le interazioni nei dibattiti online. In contrasto con il lavoro di Palau e Moens \cite{palau2009argumentation}, questo approccio mira a identificare le relazioni tra argomenti (macro-livello) e non tra i componenti dell'argomento (micro-livello).
